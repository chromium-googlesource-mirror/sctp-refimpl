\documentclass[12pt]{article}
\usepackage{cite}
\usepackage{epsfig}
\usepackage[cmex10]{amsmath}
\usepackage{fixltx2e}
\usepackage{url}
\hyphenation{op-tical net-works semi-conduc-tor}

\begin{document}
%
% paper title
%
\title{An Investigation into Data Center Congestion Control with ECN}
\author{R.~Stewart%
        \thanks{Huawei Inc.,
                2330 Central Expressway,
                Santa Clara, Ca, 95050
                USA,
                \texttt{randall@lakerest.net}.}
	 \and
	M.~T\"uxen%
        \thanks{M\"unster University of Applied Sciences,
                Department of Electrical Engineering
                and Computer Science,
                Stegerwaldstr.~39,
                D-48565 Steinfurt,
                Germany,
                \texttt{tuexen@fh-muenster.de}.}
 	\and
 	G.~Nevile-Neil%
 	\thanks{Georges Company,
	             He will take manhattan
	             New York, New York, 22222
	             USA,
	             \texttt{gnn@freebsd.org}.}
}

\maketitle


\begin{abstract}
Data centers pose a unique set of demands on any transport protocol being used
within them. It has been noted in Griffen et.al.~\cite{griffen} that common datacenter
communication patterns virtually guarantee incidents of incast. In Vasudevan et.al.~\cite{vasudevan} solving the incast
problem involved reducing the RTO.min (the minimum retransmission timeout) of the transport protocol, but
this failed to alleviate the root cause of the problem, ~switch buffer
overflow. Alizadeh et.al (DC-TCP)~\cite{alizadeh} address the same problem
 with thought given to using ECN~\cite{rfc3168}  with a new algorithm to not only eliminate
incast but to also reduce switch buffer occupancy, thus improving both elephant and mice flows
within the datacenter. In this paper we attempt to revisit some of the DC-TCP  work with a few differences, namely:

\begin{enumerate}
 \item  Instead of using only TCP~\cite{rfc793} we separate the external transport protocol from the internal datacenter protocol.
 To achieve this separation, we use SCTP~\cite{rfc4960} instead of TCP
 for the internal datacenter communication, giving us more flexibility in the feature set available to our internal datacenter, and at the same
 time assuring that changes within the transport stack internally will not adversely effect external communications on
 the Internet itself.
\item  When attempting to reproduce some of DC-TCP findings, we will use existing switch products
to provide the appropriate ECN marking.
\item Instead of using the DC-TCP algorithm we have defined a less compute intensive modification
to ECN we call Data Center Congestion Control (DCCC), implementing it within the FreeBSD SCTP stack.
\item We compare four variants of SCTP: standard SCTP, SCTP with ECN, SCTP with DCCC and an alternate
form of DCCC we call Dynamic DCCC. This version of DCCC is capable of switching between regular ECN
and DCCC based on the initial Round Trip Time (RTT) of the path. 
\end{enumerate}

\end{abstract}

% Keywords
\section*{Keywords}
Data Center Congestion Control (DCCC);
Stream Control Transmission Protocol~(SCTP);
ECN;
Transmission Control Protocol~(TCP).

\section{Introduction}
At the IETF in Beijing in the fall of 2010 M. Sridharan presented some of the DC-TCP results.
This was the original impetus that led to the development of
this paper. In reviewing their material, several questions seemed unanswered by their
work and encouraged this investigation.

\begin{enumerate}
\item Why was plain ECN not used in the DC-TCP work? ECN itself was mentioned but no
real measurements were performed contrasting DC-TCP and plain TCP with ECN.

\item The algorithm defined by DC-TCP seemed somewhat complex,  using
both floating point numbers in its calculations and also introducing
another state machine to track acknowledgement state.  Could there be a simpler method than prescribed that would obtain similar results?

\item Could existing hardware, readily available from vendors, be used to reproduce these results? The
DC-TCP paper explicitly mentions a  Broadcom Triumph,  Broadcom Scorpion, and Cisco CAT4948~\footnote{The 
CAT4948 was explicitly mentioned as NOT supporting ECN.} switch, but what is the real availability of 
a switch off the shelf to support ECN?

\item Could it be beneficial to use a separate protocol within the
  datacenter from the one used for  accessing the datacenter?
      This would allow isolating any congestion control changes within the datacenter
      while not effecting internet traffic.

\item SCTP seems to be a perfect candidate for use within a datacenter.

\begin{itemize}
\item It provides a rich set of additional features not available in
      TCP (for example it can automatically discard old data after a specified time). 

\item SCTP also provides multi-homing and CMT~\cite{jana} which could prove useful inside a datacenter.

\item SCTP's ECN implementation is more conducive to knowing when each packet is marked without
      changing any internal ECN mechanism or maintaining a small state machine to track acknowledgment state. 
      This would make implementation both simpler and more efficient.
\end{itemize}

\end{enumerate}

After considering these facts, it was decided to put together a `mini-datacenter` to simulate the
kinds of traffic highlighted in the DC-TCP work. Budget constraints meant having to spend frugally and 
using \emph{Ebay} to find used equipment that could be used to do these experiments.

\section{Gathering the equipment}

At first glance one would think gathering the necessary equipment in order to do DC-TCP like
experiments would be easy:
\begin{enumerate}
\item Find several switches that can perform ECN and configure them
to do ECN on an instantaneous basis.\footnote{Normally ECN is set to use average queue size
for its decision to mark.}

\item Setup the appropriate set of switch parameters as to when to start
marking. 

\item Add a number of servers all with the same software and test, gather results, wash, rinse, and repeat.
\end{enumerate}

Finding a switch that supports ECN proved surprisingly difficult.  
Chipsets like the Broadcom BCM56820 and BCM56630 natively support 
ECN in hardware,  but the software available from vendors that use
these chipsets did not have the the knobs to enable ECN. After looking at various CLI
configuration guides and talking with switch manufacturers, the only switch with documented
ECN support to be found was the Cisco 4500 series. Thus, instead of having several switches
to use for experimentation, only one was available. A Cat 4500 is really a "big buffer"
switch. This posed additional problems in setting up a configuration
that would allow us to do DC-TCP like tests.

Gathering the servers, however, proved to be quite easy. There are quite a number of 1U servers
for purchase on \emph{Ebay}. On average, we were able to obtain dual-core and quad-core 32-bit 
servers with two Gigabit interface at reasonable costs\footnote{For these experiments we only used
one interface but have plans to use this `datacenter' for future investigations  namely CMT.}. 

Our final lab configuration can be seen in Figure~\ref{fig:labconfig}.

\begin{figure}[h]
\centering
\epsfig{file=labconfig,scale=.45}
\caption{Final Laboratory Configuration }
\label{fig:labconfig}
\end{figure}




\subsection{Caveats when using the Cisco 4500}

We configured the switch in a way that would generate either drops
or ECN with using the Data Buffer Limiting (DBL) feature. DBL allows ECN or Random Early
Drop (RED) like features to be enabled on transmit queues within the switch.

After some initial problems finding a software version that would
support the features we wished we finally settled on an older
12.2(31)SGA9 - Enterprise Release, and this finally generated
ECN marked packets.

The DBL feature on a CAT 4500 proved less configurable than desired.There are four
basic settings that control the ECN feature. 

\begin{enumerate}

\item Maximum Credits - this value represents how many credits to issue to a
flow.\footnote{Either identified via IP addresses or IP addresses and layer 4 ports.}
The value was configurable between 1--15. A credit comes in to play only when
the switch decides it is congested. At the point the switch does decide it is
congested, it then issues this number of credits to all identified flows\footnote{In the
case where layer 4 ports are not used then a flow is considered any packets between
a source and destination IP address.}. Every additional packet a flow adds to the outgoing
transmit queue decrements the number of credits it has available. If the flow does \emph{not}
respond by reducing its rate of sending, then it may reach the point where it reaches the `Aggressive
Credits' threshold, where it will be treated as an aggressive (non-responsive) flow.


\item Aggressive Credits --- This value represents when a flow that is not responding begins
to be considered as a non-responsive flow. The aggressive credits value
can be between 1--15.

\item Aggressive Buffers --- This is the maximum number of packets a classified non-responsive
flow may have outstanding in the transmit queue. This value was configurable  between 0--255.
Any flow having more than this many packets in a queue that is considered aggressive will have
new packets dropped.

\item Exceeds Action --- This item can be set to either use ECN or not. When not
using ECN, the switch signals with a packet drop. With ECN on, flows supporting
ECN will mark the packet instead of dropping it.

\item Exceeds Action Probability --- This is a percentage of random times that the Exceed
Action will happen (i.e.~a packet will be dropped or marked).

\end{enumerate}

 When first examining the switch configuration, it appeared that just by setting some reasonable values in these
 control fields we would obtain ECN marking or drops. A first attempt was made on a Gigabit
 link using TCP\footnote{We used TCP in our initial tests to validate that the switch would do ECN since
 we were unsure, at first, if SCTP would be properly marked. Later we switched to SCTP to do all of our measurements
 after confirming that the switch would correctly mark SCTP packets.},
 but our initial incast experiments with the eleven servers we had resulted in no DBL  
 activity drops or ECN markings. Further investigation into the switch showed us that each Gigabit link has 1920 packet
 buffers dedicated to it on the transmit side\footnote{There is no configuration setting that allows us
 to lower this value.}, and the receiver side is completely non-blocking. Evidently, the
 threshold to obtain enough activity inside any given transmit queue was high enough that many more
 flows would be necessary than our small datacenter could provide with its eleven multi-core FreeBSD servers. 
 
 As an alternative, most of our servers were moved to the 100 Megabit interfaces that were also available
 on the switch. These interfaces only have 240 packets in their transmit queues, giving a more easily
 overtaken set of transmit buffers. Three of the servers,
 two eight core and one four core, were left on Gigabit interfaces to provide higher amounts of traffic
 into the remaining eight servers. With this configuration we could finally obtain DBL markings and drops
 on a reliable basis. With the help of wireshark\cite{wireshark} it was established that the switch would not
 enter a "congested state" and start marking until at least 120-130 packets were queued by a single
 flow. Two flows appeared to get the switch marking around 70-80 packets from each flow.
 
We were now left with a dilemma. In the DC-TCP experiments, the switch was configured to mark
 whenever a flow took more than N packets:  N was set to 20 for Gigabit links and 65 for
 10 Gigabit links. Our switch would not mark packets until more than
 100 packets were in the queue. Without the proper controls over when the switch would start marking packets,  how could the switch 
 be configured to approximate the circumstances used in the previous work?
 
 The solution was to configure the switch to have a maximum number
 of credits (15), the fall-over to an aggressive flow at a low value (5), the probability
 of exceeds action to 100\% and the exceeds switch buffer to the maximum to 255. With
 this configuration, marking would happen as noted earlier around 120-130 packets, and
 in the worst case (with two flows) around 75 packets, or 150 total packets outstanding. This then
 would leave somewhere between 90-120 packets available before the transmit queue would overflow.
 
 It was felt that this configuration would be closer to what could be obtained with a more
 configurable switch with less buffer space (i.e.~somewhere in the range of a switch with
 100 packet buffers available configured to start marking packets around 10).  The only downside
 to this configuration was that with 130 or more packets in queue\footnote{Before marking began the
 round trip time normally around 115us.} the round trip time would increase to up to 14-15ms.
 Though not as flexible as the configuration presented in the DC-TCP papers, it was definitely a more realistic configuration with the only
 off-the shelf  switch we could find supporting ECN.
 
\section{Targeted Communication Pattern}
In examining the results of DC-TCP it was decided to focus specifically on the interactions between
the elephant flows; in a datacenter those flows updating databases on the order of a constant one or two flows
sized randomly at about 100--500 Megabyte per flow, and the mice; partition/aggregation flows that typically 
can cause incast. In the following subsections we describe each client and server that was created for
our experiments and which is available online at \texttt{http://www.freebsd.org/\textasciitilde rrs/dccc.tar.gz}.

\subsection{Partition/Aggregation Flows}
\label{partition}

Incast occurs when simultaneous sends by a number of hosts cause buffer overflow within
a switch. For a detailed look at incast see R. Griffith et.all~\cite{griffen}. A
partition/aggregation flow is one that sends a request to multiple servers, and all of the servers respond
as rapidly as possible, usually with a small number of packets. The requestor then aggregates
this data and sends it back to the external requestor. Since most of the aggregate servers will respond
at almost the same time the partition/aggregation work model precisely fits a description of incast. 

To simulate incast on our small testbed we built a client
(\texttt{incast\_client.c}) and a server (\texttt{incast\_server.c}), 
which  do the following:
\begin{enumerate}

\item Open a socket and bind to a specific well known port and address\footnote{The multi-homing
behavior of SCTP was specifically disabled by binding to only a single address.}. After creating and
binding the socket issue a listen setting the backlog to X\footnote{We used a value of 4 for the backlog value X in our tests}.

\item Create N\footnote{N in our testbed was set to 10.} additional threads, besides the main thread.

\item Each thread, including the main thread, loops forever accepting from the main listening socket.

\item When a connection is accepted, a simple structure is read from the client connection that includes two fields, the
number of packets (or sends) to make, and the size of each send.

\item After receiving the request, the thread then sends the requested number of packets at the set
size and then closes the socket, returning to the accept call.

\end{enumerate}

The server
was always bound to the single address that our CAT 4500 was connected to the hosts over\footnote{In our case 10.1.5.X}
see Figure~\ref{fig:labconfig}.
The server also accepted one other option, whether to run over SCTP or
TCP, allowing us to use
the same server in both our early testing, validating the switches configuration for ECN, and our final tests
using SCTP.

The clients were a bit more complicated. It was decided to build a universal 
configuration file that could be reused to describe all of the peers that any
particular machine had. An example of the file from `bsd3` is shown below:

\begin{verbatim}
sctp:on
peer:10.1.5.24:0:bsd1:4:
peer:10.1.5.23:0:bsd2:4:
peer:10.1.5.22:0:bsd5:4:
peer:10.1.5.21:0:bsd6:4:
peer:10.1.5.20:0:bsd7:4:
peer:10.1.5.19:0:bsd8:4:
peer:10.1.5.18:0:bsd9:4:
peer:10.1.5.17:0:bsd10:4:
peer:10.1.5.16:0:bsd11:4:
peer:10.1.5.15:0:bsd4:8:
times:0
sends:1448
sendc:2
bind:10.1.5.14:0:bsd3:8:
\end{verbatim}

All fields within the file are separated by the \texttt{:} character.
The first line in the example contains either a \texttt{sctp:on} or
\texttt{tcp:on}, which controls the transport protocol
the incast\_client will use. 

The peer line is made up of the keyword \texttt{peer} followed by the IP address of a
machine running the server. The next field indicates the port  number the server is
listening on, where 0 indicates that the default value for the server
should be used\footnote{For the incast
client the default server port was 9902.}. The next field represents
the hostname of the peer, a value 
that is used only by some of the data processing routines. The final field represents how many bytes a
long holds; and was used by our data processing routines to determine what record format
to read\footnote{32~bit machines write the value for a time as two 4~byte longs where as 64~bit machines
write the value as two 8~byte longs.}.

The line containing the keyword \texttt{times} indicates how many times to run the test, a value of 0
means run forever.

The keyword \texttt{sends} indicates how many bytes to send. We always elected 1448 since this is
the number of bytes that will fit in a typical timestamped TCP packet, SCTP is actually capable
of fitting 1452 bytes, but it was decided to keep the same exact configuration without changing
the number of bytes in each packet.

The keyword \texttt{sendc} specifies how many packets the client should request of the incast\_server.

The final line indicates the actual address and port to bind. The format is the same as that
of a \texttt{peer} entry with the only difference being a \texttt{bind} keyword.

Once the configuration file is loaded (passed to the incast\_client program with the required
\texttt{-c} option), the process creates a kqueue, which it uses to multiplex responses from
servers. It then enters a loop for the requested number of times (which may be infinite) and
does the following:

%\newpage

\begin{enumerate}

\item Record the start time in a special header record using the precise realtime clock.

\item Build a connection to each of the peers listed in its peer list (loaded from the configuration
file). Building a connection includes opening a socket to each peer and setting NODELAY\footnote{NODELAY turns 
any Nagle\cite{rfc896} like algorithm off.} on it, connecting to each peer, and adding an entry into 
the kqueue to watch for read events from the peer.

\item After all the connection were built, another precise real time clock measurement was
taken.

\item Every server is now sent a request for data, and then its state is set to `request sent'.

\item After all the servers were sent the request, the kqueue was watched until all of the servers
responded with the prescribed number of  requested bytes. When the first byte of a
message is read\footnote{Indicated by the state being `request
  sent'.}, a precise monotonic clock time
is taken and stored with the peer entry, followed by its state being set to `reading'. When the last byte of a request arrives another 
precise monotonic clock is again taken and stored with the peer entry.

\item Once all of the peers have responded, a final precise realtime clock is taken and stored
in the header record with the number of peers.

\item The results are either displayed or stored based on
whether the user specified an output file on the command line. If the results are being
displayed, then only peers that responded in more than 300ms are shown.
If the results are being stored, the header record is written first, followed by a record for each of the
responding peers with their timings.

\item This sequence is repeated until the specified number of iterations is reached.

\end{enumerate}

\subsection{Elephant Flows}
\label{elephants}
To simulate datacenter like traffic, a small number of large flows are also needed. In a
real datacenter, these flows would provide updates to the various databases that all the query traffic is attempting to access. 

We simulated these flows
with a server~(\texttt{elephant\_sink.c}) and a client~(\texttt{elephant\_source.c}). In the simulations, each machine ran
an \texttt{elephant\_sink}; but only two machines were chosen to run the \texttt{elephant\_source}. The hosts
that ran the elephant\_source (bsd3 and bsd4) were our two, eight core, 64~bit machines, connected to the switch
via a Gigabit ethernet port. This allowed the elephant flows to always push at filling the transmit queues
of the 100 Megabit clients. One additional host (bsd8) was also connected via Gigabit ethernet to provide
an additional push on transmit queues while running the incast\_client. Note that the elephant flows that
were sent to this additional host (bsd8) were not used in measuring overall bandwidth obtained 
in our datacenter simulation.
\\

The elephant\_sink worked much like our incast\_server, creating a
number of threads, defaulting to 2, that
would do the following after binding the specified address and port numbers:
\begin{enumerate}

\item Accept a connection from a peer.

\item Record both the precise monotonic time and the precise realtime clock.

\item Read all data, discarding it but counting the number of bytes read, until the connection closed.

\item Record again the precise monotonic time and the precise realtime clock.

\item Subtract the two monotonic clock values to come up with the time of the transfer.

\item Display the results either to the user, if no output file was configured, or save
it to the output file. The results include the size of the transfer, the number of seconds and nanoseconds
that the transfer took, and the bandwidth. Note that when writing to an output file a raw binary record
was used that also included the realtime clock start and stop values.

\item Repeat this sequence indefinitely.

\end{enumerate}


The elephant\_source used the same common configuration files that were used by the incast\_client program.
After loading the configuration our elephant\_source would first
seed the random number generator (by using the monotonic precise clock), and then
do the following:

\begin{enumerate}

\item Choose a random number of bytes between 100,000,000 and
  536,870,912.\footnote{This number was arrived at as a mask to bound the
  random number since when you subtract one the number becomes 0x1fffffff}

\item Record in a header the precise realtime of the pass start.

\item Distribute the data to each peer, to do this we would connect to each peer, turning on 
NODELAY\footnote{NODELAY turns any Nagle\cite{rfc896} like algorithm off.}, record the
precise monotonic clock with the peer record, send the specified number of bytes
and close the connection. After closing the connection again, record the end monotonic 
precise time.

\item After distributing the data to all peers we would again obtain the realtime precise
clock end time for our header record.


\item Next the results would be displayed to the user or stored. When storing the
results, a header record would be written out with all of the timestamps followed
by the precise times of each of the peers.

\item These steps were then repeated until the iteration count was reached (which was
also allowed to be infinite).

\end{enumerate}

When configuring the two elephant sources, we reversed the client list order so that on a random
basis they met somewhere in their distribution attempting to send data to the same elephant\_sink at various
times during their transfers. During data analysis the elephant source file was used to first read the results, but the 
recorded timestamps of the elephant\_sink files were used to obtain the precise time of transfer (first byte
read to last byte received). Any data transfer bandwidth to either bsd3, bsd4, or bsd8 was not included in
our bandwidth graphs, since these three hosts had Gigabit interfaces which would unduly skew the
transfers.

With our pairs of clients and servers completed and tested we were almost ready to 
begin our experiments. But first we needed to examine SCTP congestion control
when handling ECN events and modify it according to our simplified algorithm.

\section{Our Data Center Congestion Control Algorithm}
\label{algo}
The algorithm used by SCTP for ECN provides a lot more information than
the standard TCP ECN. This is due to the fact that SCTP is not limited to two bits
in a header to indicate ECN and so provides a much richer environment for its
implementation. The normal response of SCTP to an ECN Echo event is the same
as TCP in any loss within a single window of data:

\begin{equation}
\begin{split}
&ssthresh = cwnd / 2\\
&if (ssthresh < mtu)~\{ \\
&~~~ssthresh = mtu; \\
&~~~RTO <<= 1\\
&\} \\
&cwnd = ssthresh
\end{split}
\label{algo1}
\end{equation}

When one or more losses occur in a single window of data, the cwnd is halved and
the ssthresh is set to the same value. When an ecn\_echo is received, the congestion
control algorithm is given two extra pieces of information besides the stcb and the 
net structures\footnote{The stcb and net structures are internal structure used within the FreeBSD SCTP
stack to track the specific SCTP association (stcb) and the specific details of a peers destination
address (net)}.

These two extra pieces of information are the keys to our new DCCC algorithm, they are
the in\_window flag as well as the number of packets marked. The in\_window flag
is used by the current algorithm to know if it should perform the
algorithm shown in Equation [1],  and
the number of packets marked is ignored by the existing algorithm. For
our new algorithm we use both of these values combined with another value passed
to us via the network structure (net).

Each time a SCTP chunk is sent in a packet, we also record the congestion
window at the time the transmission was made\footnote{The value is recorded
on an internal structure used to track the chunk.}. When an ECN Echo arrives, before calling
the modular congestion control, an attempt to find the
TSN\footnote{A Transport Sequence Number is the unit of message transmission that
SCTP uses when it sends data.}  being reported is made. The TSN
will always be found if the ECN Echo is in a new window of data. If the TSN
is found, the previous cwnd at the time that the TSN was sent\footnote{Found on the chunk structure mentioned earlier}, is recorded on
the net structure in a special field for ECN (ecn\_prev\_cwnd).  These three values are then used as shown in Equation [2].

\begin{equation}
\begin{split}
if (&in\_window == 0)~\{ \\
&cwnd = ecn\_prev\_cwnd - (mtu * num\_pkt\_marked) \\
&ssthresh = cwnd - (num\_pkt\_marked * mtu) \\
\}~e&lse~\{\\
&ssthresh -= mtu * num\_pkt\_marked \\
&cwnd -= mtu * num\_pkt\_marked\\
\}~~&
\end{split}
\end{equation}

It is important to remember when looking at SCTP congestion control
parameters that there
is a separate set of parameters for each destination address (RTO, cwnd, ssthresh and mtu). In
our experiments this was never an issue since SCTP was kept in strict single-homed mode
by binding explicitly a single address. 

As noted earlier we also used a third algorithm. We termed this version dynamic DCCC, because the algorithm would switch between normal ECN behavior and DCCC based on the
initial round trip time measured on the first data packet. SCTP uses this value to try to determine
if the destination address is on a local LAN. If it thinks that it is on a local LAN (i.e.~having a round
trip time of under 1.1ms), then it sets a flag. Our dynamic algorithm simply used
that flag to switch between the algorithms shown in Equation [1] and Equation [2].

\section{Measurements}
\label{measure}
We first decided to examine incast and how the three algorithms (DCCC, DYN-DCCC and ECN) compared with
running plain SCTP in our network. We started an incast\_server and elephant\_sink on each machine (bsd1--bsd11) and then 
started an incast\_client passing it options to start collecting data. Once all incast servers were running, we started two elephant sources one on bsd3 the other on bsd4. With the switch configured to drop, i.e.~not do ecn, we
then record all results and were able to obtain the plots show in Figure~\ref{fig:noEcnIncast}\footnote{Time below 15ms excluded}.
\begin{figure}[h]
\centering
\epsfig{file=noEcnIncast,scale=.95}
\caption{Normal SCTP transfers experiencing incast}
\label{fig:noEcnIncast}
\end{figure}

In Figure~\ref{fig:noEcnIncast} we see incast occurring in cases where the time to complete a transfer exceeds
300ms\footnote{The default RTO.min timeout value for SCTP}. In some cases we see more than
one retransmission. Clearly incast is occurring on a fairly regular basis. Counting the actual number
of incast occurrences shows us 567 different incidents of incast  represented in Figure~\ref{fig:noEcnIncast}.

\newpage

Next it was decided to turn on ECN in the switch with no changes to the congestion control algorithm. This
lead to the results seen in Figure~\ref{fig:ecnIncast}\footnote{Time below 15ms excluded}.

\begin{figure}[h]
\centering
\epsfig{file=ecnIncast,scale=.95}
\caption{Normal SCTP with ECN experiencing incast}
\label{fig:ecnIncast}
\end{figure}


Note that simply enabling ECN on the switch with our settings reduces the incidents of incast to a minimal
level. There are only 4 occurrences of incast that reach the 300ms level. This is a vast improvement over
our previous figure.  

Curiously,  bsd4 shows quite a few delays in returning results, below the incast level, 
but still quite high. Since these measurements are taken from the perspective of the aggregator and
bsd4 is connected via Gigabit ethernet, clearly it must be caused by delays in the incast server
receiving the initial request to transfer the two packets back to it. Most likely this is caused by
a lost setup message during the connection startup.

\newpage

Next, the same tests were run, but this time using our DCCC algorithm in place in the SCTP stack. The
results from this run can be seen in Figure~\ref{fig:dcccIncast}\footnote{Time below 15ms excluded}.

\begin{figure}[h]
\centering
\epsfig{file=dcccIncast,scale=.95}
\caption{DCCC ECN experiencing incast}
\label{fig:dcccIncast}
\end{figure}


Surprisingly, the actually occurrence of incast increases in this graph to 34 separate incidents. This
was unexpected, but with a look at some of our further graphs we will see possible
reasons why so much unexpected incast is occuring.

\newpage

Finally, we ran the tests again only this time we used our Dynamic DCCC algorithm. This allowed the
SCTP stack to switch between normal ECN and DCCC ECN based on the round trip time
seen when transfer of data first begins. The results of this test can be seen in Figure~\ref{fig:dynIncast}\footnote{Time below 15ms excluded}.


\begin{figure}[h]
\centering
\epsfig{file=dynIncast,scale=.95}
\caption{Dynamic DCCC ECN experiencing incast}
\label{fig:dynIncast}
\end{figure}

Here we see no incidents of incast over 300ms, which is an unexpected
outcome since we had anticipated no differences between DCCC (with its 34 incidents) and
the dynamic version. If we turn, however, to the elephant graphs, we can reach a conclusion
as to why these incidents are occurring.

\newpage

Lets look at the four graphs of the elephant transfers that were being run while all
the incast was occurring. Figure~\ref{fig:noEcnBw} shows a normal SCTP transfer with
ECN disabled. Note the huge swings downward in throughput for both
flows when the flows collide and share transfer to the same machine. 

During these incidents, we can be sure that a large number of drops are occurring. A look at the switch
information tells us that not only are DBL drops happening but also tail
drops from transmit queue overflows are occurring as well.  The switch reports 19,312 DBL drops
and 46 Transmit queue drops. The SCTP statistics show
a large number of T3-Timeouts\footnote{A T3-Timeout, in SCTP, is the firing of the SCTP data retransmission timer.}
 i.e.~1360. Any T3 timeout in the incast test would definitely
cause an incident of incast since there is not enough data in the pipe to enable fast retransmits with
only two packets being sent.

\begin{figure}[h]
\centering
\epsfig{file=noEcnBw,scale=.95}
\caption{Bandwidth of two flows no ECN}
\label{fig:noEcnBw}
\end{figure}


\newpage

Figure~\ref{fig:ecnBw} show the data transfer when ECN is enabled on
the switch but no alternate congestion control is running. In this graph,
we see fewer swings and a more orderly set of transfers. Checking
the switch error counts we find that there occurred 1,079 DBL drops and 65 tail
drops. SCTP's statistics also show 43 T3-Timeouts\footnote{A T3-Timeout, in SCTP, is the firing of the SCTP data retransmission timer.},
but far from the larger number seen in the instance of no ECN.

\begin{figure}[h]
\centering
\epsfig{file=ecnBw,scale=.95}
\caption{Bandwidth of two flows with ECN}
\label{fig:ecnBw}
\end{figure}


\newpage

We enable our modified congestion control with DCCC, and the bandwidth
results are seen in Figure~\ref{fig:dcccBw}, showing similar behavior to ECN but with a bit better performance in
overall bandwidth. The error counters on the switch, however,
show 1,890 DBL drops and 1,441 tail drops. This tells us that  the more aggressive congestion
control, when the two large flows meet on the same host, end up overflowing the switch
transmit queues. The SCTP statistics show similar results of 422 
T3-Timeouts\footnote{A T3-Timeout, in SCTP, is the firing of the SCTP data retransmission timer.}. 


\begin{figure}[h]
\centering
\epsfig{file=dcccBw,scale=.95}
\caption{Bandwidth of two flows with DCCC}
\label{fig:dcccBw}
\end{figure}


\newpage

Finally, the bandwidth of our Dynamic DCCC is shown in Figure~\ref{fig:dynBw}. Clearly the
overall bandwidth is better than in any other of the tests. The switch error counters show that only 35 tail
drops occurred and 1,028 DBL drops. SCTP statistics show only 42 T3-Timeouts\footnote{A T3-Timeout, in SCTP, is the firing of the SCTP data retransmission timer.}. 
This would explain the lack of incast on our earlier incast chart for Dynamic DCCC.

\begin{figure}[h]
\centering
\epsfig{file=dynBw,scale=.95}
\caption{Bandwidth of two flows with Dynamic DCCC}
\label{fig:dynBw}
\end{figure}


You can see that the large flow that arrives first on a particular
host is being more aggressive, while the later arriving large flow is less aggressive. This is due to
the fact that our switch will not start marking until 130 packets. At about 115us per packet this means
that the round trip time must be close to 14ms by the time the second flow arrives. This pushes that
flow into standard ECN mode. The combination obviously keep both flows aggregate packet load
offered much smaller and thus less drops occur.

One thing that is not illustrated in any Cisco documentation is what a DBL drop means. Considering
that we have the exceeds action set at 255 packets, 15 packets above the transmit queue size, one
would expect that you would only get tail drops occurring.  It is important to also remember that a DBL 
drop would occur for any non-Data packet since the ECN code-point ECT0\footnote{ECT0 is the
mark used in the IP level header by a transport protocol to indicate ECN support to a switch
or router.} would NOT be enabled on any packet that
does not contain data. Thus our DBL drops could be lost SACK's or other connection setup packets.
Without further information from Cisco, it is hard to tell if the DBL drop counter is strictly drops or
if it also includes the number of ECN marked packets that would have been dropped.


Finally, Figure~\ref{fig:aggBw}  shows us an aggregate bandwidth comparison of the 
overall throughput of the 4 different tests. Each of the two flows in the test are summed
together to give an overall aggregate and then these are summed to give us the average
goodput of the combined flow on a per second basis. 

\begin{figure}[h]
\centering
\epsfig{file=agg,scale=.95}
\caption{Aggregate Bandwidth Comparison}
\label{fig:aggBw}
\end{figure}

As expected the ECN flow was better than the non-ECN flow by approximately 2\%. The
pure DCCC flow increased its bandwidth by nearly 5.5\% over the standard SCTP with no ECN.
And as expected from our earlier graphs, the Dynamic DCCC yielded close to a 7.9\% bandwidth
improvement over standard SCTP. As can be seen from the graph, this is quite a distance from the
theoretical maximum. Note that the theoretical maximum includes a key assumption i.e. that 
no two flows would be transferring to the same computer at the same time. This could only
be achieved by flow coordination between the two elephant flows.

\section{Conclusion and Future work}

After examining the data presented here, packet traces, and respective
graphs from those traces\footnote{Not shown for brevity, but tools to generate them from a wireshark trace are included in the source
code drop for those wanting to recreate our results}, the most clear conclusion reached is that 
with \emph{proper} configuration, enabling ECN will virtually eliminate incidents of
incast in the datacenter. This is, of course, provided that the SCTP or
TCP stack properly implements ECN within it. 
\\

So what should switch vendors take away from these results?

\begin{enumerate}

\item They should have an implementation of ECN for their datacenter
  switches, and having one
for non-datacenter switches could also be a benefit to their products. Many of the more modern switch an a chip's (e.g.~BCM56820)
already support ECN in hardware and its just a matter of allowing its users access to those
hardware features.

\item Vendors need to allow configuration for both `network level settings' (i.e.~those classically called
for by ECN) and `Datacenter settings'. Network level settings would allow the switch to use average queue
size and hopefully set thresholds to start marking and dropping in terms of percentage of queue occupied by
a flow. Datacenter settings would allow a switch to be configured for observing instantaneous queue size
and allow control of the number of packets to start marking and
dropping at, instead of percentages of
queue size.

\item Vendors need to make sure that all transport protocols can be fully supported that use ECN. This
means, if a vendor supports the concept of a flow, it needs to not be restricted to just TCP, but also should include SCTP. Using the same port
numbers with SCTP would be acceptable, but innovative providers could also use SCTP v-tags for this purpose.\footnote{Our observations
never determined if the CAT4500 considered all traffic between IP address or individual SCTP flows since there was
no way to determine if the CAT4500 understood SCTP.}

\end{enumerate}

Other conclusions from this work are much less clear. The performance gain of the various DCCC algorithms
was evident but these gains were limited (5.5 and 7.9\%). Clearly this is an advantage, but is it an advantage
worth changing congestion control algorithms within the transport protocol? The dynamic version is obviously most
attractive since if it was implemented, it could easily exist on the
internet with no impact, since it would switch itself
off in cases of a longer RTT time\footnote{We used 1.1ms but would think something smaller, perhaps 700us would
be better for general deployment}. 

In thinking of our limited switch configuration, if this was changed so that marking could begin at 5--10 packets
would we see the same results? Clearly the dynamic DCCC would become less dynamic and might look more
like plain DCCC. But would plain DCCC do better since it would no longer be hitting tail drops 
by overrunning the switch transmit queues? Also the plain SCTP-ECN variant may well perform less well 
in a switch that was more aggressively marking ECN. These questions are some of the future work that
is envisioned to attempt with a more controllable switch. The authors have been informed that Broadcom
offers a development kit which hopefully can be acquired at a reasonable cost and configured in such
a way as to provide insights into these questions.
     

\section*{Acknowledgment}

Thanks to  Randall L. Stewart and Anne Stewart who have been inflicted upon by their father for their comments, and special thanks to our editor,  Sandra Stewart.

\section*{Appendix A - Switch Configuration}

For those interested in repeating our experiments, we show the Cisco CAT4500 CLI
configuration for our switch.  Note that we varied the QOS DBL exceed action between
ECN and no ECN.

\subsubsection{QOS DBL CONFIGURATION}
 
Here we show the basic QOS DBL configuration used. It was
also not ever certain that the CAT 4500 could actually distinguish SCTP ports in separate flows.\footnote{This
may mean that the switch always saw traffic from each machine as a single flow, but this will have no impact on
our results since all measurements would be impacted (ECN and non-ECN) in the same way.}
\begin{verbatim}
no qos dbl flow include vlan
qos dbl exceed-action ecn
qos dbl exceed-action probability 100
qos dbl credits aggressive-flow 5
qos dbl buffers aggressive-flow 255
qos dbl   
qos
\end{verbatim}
A show qos dbl on our CAT 4500 displays the following:
\begin{verbatim}
QOS is enabled globally
DBL is enabled globally
DBL flow does not include vlan
DBL flow includes layer4-ports
DBL uses ecn to indicate congestion
DBL exceed-action probability: 100%
DBL max credits: 15
DBL aggressive credit limit: 5
DBL aggressive buffer limit: 255 packets
\end{verbatim}

\subsubsection {POLICY MAP AND CLASS MAP}

Here we show the specific class map and policy map used. Notice
the interfaces apply a  transmit to shape traffic to one Gigabit. This was suggested
by a former Cisco colleague to get ECN and DBL to work. The settings
are high enough for both our 100 Megabit and Gigabit links so that 
the shaping action would never be applied to the interface. Only the
QOS DBL actions would influence the transmit queue. Shaping actions
on a Cisco 4500 happen before DBL actions.
\begin{verbatim}
class-map match-all c1
 match any 

policy-map p2
 class c1 
  dbl  
  police 1000 mbps 100 kbyte conform-action transmit exceed-action drop 
\end{verbatim}
\subsubsection {Example Interface Configuration}

Here is an example of our specific interface configurations. Notice that 
for Gigiabit interfaces we turned off flow control in the initial attempts at
trying to use these interface to get ECN markings.
\begin{verbatim}
interface FastEthernet2/48
service-policy output p2

interface GigabitEthernet3/1
  flowcontrol receive off
  service-policy output p2
\end{verbatim}


\section*{Appendix B - Whats in the tarball?}
Here is a brief list the program files you will find in the tarball, if you download it. Many of
these utilities were used to generate data not presented here, but they may be of interest
to those attempting to recreate the results. All software is released under BSD license.


\begin{description}

\item[display\_ele\_client.c] This program understands how to access stored elephant results file from a
client and will further look for timing from the stored elephant\_sink output results. A common naming
scheme is used to find the files, and the store option really takes a directory path prefix. 

\item[display\_ele\_sink.c] This program knows how to read a single elephant sink file.

\item [display\_inc\_client.c] This program has the ability to read and interpret the incast client output.

\item [ele\_aggregator.c] This utility was used to aggregate multi-machine elephant output.

\item [elephant\_sink.c] This utility was described earlier and is the actual measurement tool for elephant flows.

\item [elephant\_source.c] This utility was described earlier and contains the client code that drives the elephant
flows.

\item [incast\_client.c] This is the incast generation program described earlier.

\item [incast\_lib.c] This is the common library utilities that most of these programs used.

\item [incast\_server.c] This is the incast server described earlier.

\item [per\_aggregator.c] This utility is yet another aggregator used in summing data.

\item [read\_cap.c] This utility is a special pcap reader capable of reading a tcpdump\footnote{providing the -s0 option
is used with the tcpdump} or tshark dump. It analyzes SCTP flow information and can create output for gnuplot
to observe packets outstanding as well as packets outstanding when ECN Echo's occur.

\item [sum\_aggregator.c] This is yet another aggregation utility.

\end{description}

\begin{thebibliography}{9}

\bibitem{griffen}R. Griffith et.al.,
  ``Understanding TCP incast throughput collapse in datacenter networks'', 
  In WREN Workshop, 2009.
  
  \bibitem{vasudevan}V. Vasudevan et. al.,
  ``Safe and effective fine-grained TCP retransmissions for datacenter communication.''
 In SIGCOMM, 2009.
 
 \bibitem{alizadeh}M. Alizadeh et. al.,
 ``Data Center TCP (DCTCP)'',
 In SIGCOMM, 2010.
 
\bibitem{rfc3168}  K~Ramakrishnan, S~Floyd, D~Black,
  ``The Addition of Explicit Congestion Notification (ECN) to IP'',
  \textit{RFC~3168},
  September~2001.
 
\bibitem{rfc793}  J~Postel,
  ``Transmission Control Protocol'',
  \textit{RFC~793},
  September~1981.
  
\bibitem{rfc4960} R.~Stewart,
  ``Stream Control Transmission Protocol'',
  \textit{RFC~4960},
 September~2007.
 
 \bibitem{jana}J Iyengar et. al.,
 ``Concurrent Multipath Transfer Using SCTP Multihoming'',
 In SPECTS, 2004.
 
 \bibitem{wireshark}M Scalisi,
 ``Track down network problems with Wireshark'',
 On line at http://www.computerworld.com/s/article/9144700/Track\_down\_network\_problems\_with\_Wireshark.
 
 \bibitem{rfc896}  J~Nagle,
 ``Congestion Control in IP\/TCP Internetworks'',
  \textit{RFC~896},
 January~1984.
 
\end{thebibliography}




% that's all folks
\end{document}


